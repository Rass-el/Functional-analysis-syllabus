\chapter{Понятия о функциональных пространствах.
 Некоторые понятия общей топологии}
\section{Общая топология}
\subsection{Определение топологического пространства}

Пусть X — некоторое множество. Рассмотрим набор $\mathcal{T}$ его подмножеств, для которого: 
\begin{enumerate}
    \item $\varnothing, X \in \mathcal{T}$
    \item Объединение любого семейства множеств, принадлежащих совокупности $\mathcal{T}$, также принадлежит совокупности $\mathcal{T}$
    \item Пересечение любого конечного семейства множеств, принадлежащих совокупности $\mathcal{T}$, также принадлежит совокупности $\mathcal{T}$
\end{enumerate}
В таком случае: 
\begin{enumerate}
    \item $\mathcal{T}$ есть топологическая структура или просто топология в множестве X; 
    \item   множество X с выделенной топологической структурой $\mathcal{T}$ (т.е. пара $(X, \mathcal{T})$) называется топологическим пространством;
    \item   элементы множества X называются точками этого топологического пространства; 
    \item   элементы множества $\mathcal{T}$ называются открытыми множествами пространства $(X, \mathcal{T})$. 
\end{enumerate}
Примеры топологий: \\
$\mathcal{T} = \{\varnothing, X\}$ - тривиальная топология. \\
$\mathcal{T} = 2^X$ - дискретная топология. \\ 
$F = \overline{G} = X \backslash G \in \mathcal{T}$ - замкнутое множество.\\
Для класса замкнутых множеств выполняются все три аксиомы топологии:
\begin{enumerate}
    \item $\varnothing, X \in \mathcal{T}$
    \item $\bigcup\limits_{j=1}^p G_i \in \mathcal{T}$
    \item $\bigcap\limits_{\alpha \in A} G_\alpha \in \mathcal{T}$ 
\end{enumerate}
\newpage
\noindent \textbf{Окрестность в топологическом пространстве} \\
$\forall x \in X, O(x) = \{ U \ | \ \exists G \in \mathcal{T}: x \in G \subset O(x) \}$ 
\\
\textbf{Предельная точка множества} 
\\ 
Пусть $A \subset X, b \in X $ и в любой $O(b)$ содержится хотя бы одна точка из $A$. 
Тогда $b$ - предельная точка множества $A$. 
Cовокупность всех предельных точек $A$ называется \textbf{замыканием} множества $A$ и обозначается как $Cl \ A$.
\\
Легко понять, что замыкание множества является пересечением всех замкнутых
множеств, содержащих это множество, то есть 
\\
$Cl \ A = \bigcap\limits_{A \subset F} F$ 
\\
\textbf{Внутреняя точка} множества A - точка, которая содержится в множестве A
вместе с некоторой окрестностью. \\
\textbf{Внутренностью} множества, лежащего в топологическом пространстве,
называется наибольшее по включению открытое множество, содержащееся
в нем. Внутренность множества A обозначается символом $Int \ A$ 
\\
Всякое подмножество топологического пространства обладает внутренностью. Ею
является объединение всех открытых множеств, содержащихся в этом 
множестве. То есть 
\\
$Int \ A = \bigcup\limits_{G \subset A}G$, где $G \in \mathcal{T}$ 
\\
Пример: \\
Возьмем $\mathbb{R}$ с канонической (стандартной) топологией $\mathcal{T}$.
Пусть $A =[a, b] \cap \mathbb{Q}$.
Тогда $Cl \ A = [a, b]$, а
$Int \ A = \varnothing$ 
\\
$Cl \ A \backslash Int \ A = Fr \ A$ - \textbf{граница} множества А.
\subsection{Предельный переход}
$\{x_n\} \in X, x = \lim\limits_{n \to \infty} x_n <=> \forall O(x) \ \exists N : \forall n > \mathbb{N} => x_n \in O(x)$
\\
Если $x$ - предел последовательности точек из А, то эта точка - предельная
точка множества А.
\\
В общем случае операцию замыкания нельзя определить секвенциально, то
есть на языке последовательностей.
\subsection{Непрерывность отображения}
$f\colon (X, \mathcal{T}) \to (X', \mathcal{T})$
\\
Говорят, что $f$ непрерывна в $x_0 \in X$, если $\forall O(f(x_0)) \ \exists O(x_0) : \forall x \in O(x_0) => f(x) \in O(f(x_0))$ 
\\
Если $f$ непрерывна в каждой точке А, то $f$ непрерывна на А. 
\\
Легко проверить, что отображение непрерывно на Х, когда прообраз открытого
множества в Х' открыт в Х.
\subsection{Относительная топология}
$(X, \mathcal{T}), \ E \subset X, \ \mathcal{T}_E = \{G \cap E, \ G \in \mathcal{T}\}$
\\
$\mathcal{T}_E$ - \textbf{относительная топология} в $E$.
\subsection{База топологии}
Часто топологическую структуру задают посредством описания 
некоторой ее части, достаточной для восстановления всей структуры.
\textbf{Базой} топологии называется некоторый набор открытых множеств,
такой, что всякое непустое открытое множество представимо в виде
объединения множеств из этого набора. К примеру, всевозможные интервалы
составляют базу топологии вещественной прямой.
\newpage
\textbf{Теорема 1}: Пусть $X$ - абстрактноое множество. $\sigma$ - совокупность 
его подмножеств.
\\
$\sigma$ является базой некоторой топологии тогда и только тогда, когда 
выполнены следующие условия: 
\begin{enumerate}
    \item Для любой точки $x \in X$ найдется $B \in \sigma$ такое, что $x \in B$
    \item $B_1, B_2 \in \sigma => \forall b \in B_1 \cap B_2 \ \exists
    B \in \sigma: \ b \in B \subset B_1 \cap B_2$
\end{enumerate}
Эта топология строится из всевозможных объединений 
множеств из $\sigma$, то есть \\
$\mathcal{T} = \{G=\bigcup\limits_{\alpha, \beta \in A}B_\alpha, B_\beta \in \sigma\}$
\\
\textbf{Доказательство} \\
Необходимость: \\
1) Всё множество-носитель $X$, будучи открытым множеством в $\mathcal{T}$, представимо 
в виде объединения некоторых множеств из базы. \\
Докажем выполнение условия 2): заметим, что любое множество 
из базы само открыто, а поэтому пересечение двух таких 
множеств снова открыто и должно быть представимо в виде
объединения некоторых множеств из базы. Среди этих последних
и найдётся то, которое содержит точку $x$.
\\
Достаточность: \\
1) аксиома очевидно выполняется. \\
2) аксиома очевидно выполняется.  \\
Докажем выполнение аксиомы 3): Пересечение двух открытых множеств открыто.
Пусть $G_1 = \bigcup\limits_\alpha B_\alpha, \ 
G_2 = \bigcup\limits_\beta B_\beta$. Тогда $G_1 \cap G_2 = \bigcup\limits_{\alpha, \beta}(B_\alpha \cap B_\beta)$.
Но из условия 2) следует, что каждое $B_\alpha \cap B_\beta$ содержится в $\mathcal{T}$, а тогда и $\bigcup\limits_{\alpha, \beta}(B_\alpha \cap B_\beta)$
содержится в $\mathcal{T}$. ч.т.д.
\subsection{Первая и вторая аксиомы топологии}
Если у любой точки топологического пространства есть счетная база
ее отрестностей, то это пространство удовлетворяет \textbf{1-ой аксиоме
счетности}. 
\\
Если у топологического пространства есть база топологии, состоящая
из счетного чила множеств, то оно удовлетворяет \textbf{2-ой аксиоме счетности}.