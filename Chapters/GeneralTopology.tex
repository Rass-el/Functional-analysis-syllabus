\chapter{Понятия о функциональных пространствах.
 Некоторые понятия общей топологии}
\section{Общая топология}
\subsection{Определение топологического пространства}

Пусть X — некоторое множество. Рассмотрим набор $\mathcal{T}$ его подмножеств, для которого: 
\begin{enumerate}
\item $\varnothing, X \in \mathcal{T}$
\item Объединение любого семейства множеств, принадлежащих совокупности $\mathcal{T}$, также принадлежит совокупности $\mathcal{T}$
\item Пересечение любого конечного семейства множеств, принадлежащих совокупности $\mathcal{T}$, также принадлежит совокупности $\mathcal{T}$
\end{enumerate}
В таком случае: 
\begin{enumerate}
\item $\mathcal{T}$ есть топологическая структура или просто топология в множестве X; 
\item   множество X с выделенной топологической структурой $\mathcal{T}$ (т.е. пара $(X, \mathcal{T})$) называется топологическим пространством;
\item   элементы множества X называются точками этого топологического пространства; 
\item   элементы множества $\mathcal{T}$ называются открытыми множествами пространства $(X, \mathcal{T})$. 
\end{enumerate}
Примеры топологий: \\
$\mathcal{T} = \{\varnothing, X\}$ - тривиальная топология. \\
$\mathcal{T} = 2^x$ - дискретная топология. \\ \\
$F = \overline{G} = X \backslash G \in \mathcal{T}$ - замкнутое множество.
