\documentclass[a4paper,12pt]{report} %размер бумаги устанавливаем А4, шрифт 12пунктов
\usepackage[T2A]{fontenc}
\usepackage[utf8]{inputenc}%включаем свою кодировку: koi8-r или utf8 в UNIX, cp1251 в Windows
\usepackage[english,russian]{babel}%используем русский и английский языки с переносами
\usepackage{amssymb,amsfonts,amsmath,mathtext,cite,enumerate,float} %подключаем нужные пакеты расширений
\usepackage[dvips]{graphicx} %рисунки
\graphicspath{{images/}}%путь к рисункам

\makeatletter
\renewcommand{\@biblabel}[1]{#1.} % Заменяем библиографию с квадратных скобок на точку:
\makeatother

\usepackage{geometry} % Меняем поля страницы
\geometry{left=2cm}% левое поле
\geometry{right=1.5cm}% правое поле
\geometry{top=1cm}% верхнее поле
\geometry{bottom=2cm}% нижнее поле

\renewcommand{\theenumi}{\arabic{enumi}}% Меняем везде перечисления на цифра.цифра
\renewcommand{\labelenumi}{\arabic{enumi}}% Меняем везде перечисления на цифра.цифра
\renewcommand{\theenumii}{.\arabic{enumii}}% Меняем везде перечисления на цифра.цифра
\renewcommand{\labelenumii}{\arabic{enumi}.\arabic{enumii}.}% Меняем везде перечисления на цифра.цифра
\renewcommand{\theenumiii}{.\arabic{enumiii}}% Меняем везде перечисления на цифра.цифра
\renewcommand{\labelenumiii}{\arabic{enumi}.\arabic{enumii}.\arabic{enumiii}.}% Меняем везде перечисления на цифра.цифра

\begin{document}
\newtheorem{theorem}{Теорема}

\begin{titlepage}
    \newpage
    
    \begin{center}
    \LARGE Университет ИТМО \\
    \vspace{1cm}
    \normalsize Кафедра пивоварения \\
    \hrulefill
    \end{center}
     
    \flushright{Элитное подразделение \\ Университета ИТМО}
    
    \vspace{8em}
    
    \begin{center}
    \huge Конспект лекций по \\ функциональному анализу
    \end{center}
    
    \vspace{2.5em}
     
    \begin{center}
    \textsc{\textbf{Додонова Николая Юрьевича}}
    \end{center}
    
    \vspace{6em}
     
    
     
    \vspace{\fill}
    
    \begin{center}
    Санкт-Петербург 2к19 
    \end{center}
    
\end{titlepage}% это титульный лист
\tableofcontents % это оглавление, которое генерируется автоматически
\chapter{Понятия о функциональных пространствах.
 Некоторые понятия общей топологии}
\section{Общая топология}
\subsection{Определение топологического пространства}

Пусть X — некоторое множество. Рассмотрим набор $\mathcal{T}$ его подмножеств, для которого: 
\begin{enumerate}
\item $\varnothing, X \in \mathcal{T}$
\item Объединение любого семейства множеств, принадлежащих совокупности $\mathcal{T}$, также принадлежит совокупности $\mathcal{T}$
\item Пересечение любого конечного семейства множеств, принадлежащих совокупности $\mathcal{T}$, также принадлежит совокупности $\mathcal{T}$
\end{enumerate}
В таком случае: 
\begin{enumerate}
\item $\mathcal{T}$ есть топологическая структура или просто топология в множестве X; 
\item   множество X с выделенной топологической структурой $\mathcal{T}$ (т.е. пара $(X, \mathcal{T})$) называется топологическим пространством;
\item   элементы множества X называются точками этого топологического пространства; 
\item   элементы множества $\mathcal{T}$ называются открытыми множествами пространства $(X, \mathcal{T})$. 
\end{enumerate}
Примеры топологий: \\
$\mathcal{T} = \{\varnothing, X\}$ - тривиальная топология. \\
$\mathcal{T} = 2^x$ - дискретная топология. \\ \\
$F = \overline{G} = X \backslash G \in \mathcal{T}$ - замкнутое множество.

\chapter{Метрические пространства}
\chapter{Нормированные пространства}
\section{Нормированное пространство}
\section{Баноховое пространства}
\section{Теорема Риса}
\chapter{Евклидовые пространства}
\end{document}